% Year 1 Section 2: Methods, procedures, and implementation schedule
% Sessions B + C per CM03_production_spec.md

\noindent\textbf{(1) Research principles, methods, and the innovation of research methods}

\bigskip

\noindent\textbf{(1.1) The OI-SVMVAR Framework}

\medskip

To address the identification challenges outlined in Section~1, this project employs the \textbf{order-invariant stochastic volatility in mean vector autoregression} (OI-SVMVAR) framework developed by \citet{davidson2025investigating}. We select this framework because it uniquely resolves the three methodological obstacles that confront any attempt to trace external uncertainty shocks through a small open economy: it accommodates a large information set, eliminating the omitted variable bias that confounds small-scale models; its order-invariant identification ensures that results do not depend on arbitrary variable ordering; and its time-varying classification mechanism allows the data---rather than the researcher---to determine whether a given external shock transmits through macroeconomic or financial channels. The model consists of five coupled components, estimated jointly in a single Bayesian step, which we now present in turn.

To capture the direct impact of uncertainty on real economic activity and financial conditions, the observation equation embeds the latent uncertainty factors in the conditional mean of a VAR:
%
\begin{equation}
\mathbf{y}_t = \mathbf{c} + \sum_{l=1}^{p} \mathbf{A}_l\, \mathbf{y}_{t-l} + \boldsymbol{\Gamma}\, \mathbf{h}_t + \boldsymbol{\varepsilon}_t,
\label{eq:var}
\end{equation}
%
where $\mathbf{y}_t$ is the $N \times 1$ vector of observables (in our application, $N = 43$ variables spanning Taiwan's domestic economy, U.S.\ and Chinese indicators, and global risk measures); $\mathbf{c}$ is an $N \times 1$ intercept vector; $\mathbf{A}_l$ are $N \times N$ coefficient matrices at lag $l = 1, \ldots, p$; $\boldsymbol{\varepsilon}_t$ is the $N \times 1$ reduced-form disturbance vector; and $\mathbf{h}_t = (h_{m,t},\, h_{f,t})'$ is a $2 \times 1$ vector containing the two latent common log-volatility factors that represent Taiwan's \textbf{macroeconomic uncertainty} ($h_{m,t}$) and \textbf{financial uncertainty} ($h_{f,t}$), respectively. The $N \times 2$ matrix $\boldsymbol{\Gamma}$ captures the \textbf{stochastic volatility in mean} effect: because $\boldsymbol{\Gamma}\, \mathbf{h}_t$ enters the conditional mean of $\mathbf{y}_t$, elevated uncertainty directly shifts the expected path of all observables---output, employment, asset prices, and credit conditions alike. This is what distinguishes the SVMVAR from a standard stochastic volatility model, in which volatility appears only in the error covariance and thus cannot feed back into the levels of economic variables. The inclusion of $\boldsymbol{\Gamma}\, \mathbf{h}_t$ allows us to test the prediction---central to real-options and precautionary-savings theories \citep{bloom2009impact, bloom2014fluctuations}---that heightened uncertainty depresses investment, consumption, and hiring not merely by increasing dispersion but by shifting the conditional mean of economic outcomes.

To ensure that identification does not depend on the arbitrary ordering of variables within $\mathbf{y}_t$, the contemporaneous relationship between reduced-form and structural disturbances is specified as
%
\begin{equation}
\mathbf{B}_0\, \boldsymbol{\varepsilon}_t = \boldsymbol{\Sigma}_t^{1/2}\, \mathbf{u}_t, \qquad \mathbf{u}_t \sim \mathcal{N}(\mathbf{0},\, \mathbf{I}_N),
\label{eq:contemp}
\end{equation}
%
where $\mathbf{B}_0$ is an $N \times N$ contemporaneous impact matrix and $\boldsymbol{\Sigma}_t = \mathrm{diag}(\sigma_{1,t}^2, \ldots, \sigma_{N,t}^2)$ is the diagonal matrix of time-varying, variable-specific conditional variances. The structural disturbances $\mathbf{u}_t$ are homoskedastic by construction; all time variation in volatility is absorbed by $\boldsymbol{\Sigma}_t$. Crucially, $\mathbf{B}_0$ is left as a full, unrestricted nonsingular matrix---subject only to the normalisation that its diagonal elements equal one---rather than being constrained to a lower triangular form. This unrestricted specification is what delivers \textbf{order invariance}: because no zero restrictions are imposed on the off-diagonal elements of $\mathbf{B}_0$, the estimation results are identical regardless of how variables are ordered in $\mathbf{y}_t$. The implications of this design choice---and why it is indispensable for a 43-variable system that mixes domestic and external series---are discussed in detail in subsection~(1.3).

The structural heart of the model lies in the specification linking each variable's log-volatility to the common uncertainty factors. Following \citet{davidson2025investigating}, the $N$ variables in $\mathbf{y}_t$ are partitioned into three blocks---$n_m$ macroeconomic variables, $n_f$ financial variables, and $n_u$ unclassified variables, with $n_m + n_f + n_u = N$---and the log conditional variance of variable $i$ is decomposed into a \textbf{variable-specific idiosyncratic component} $\eta_{i,t}$ and a loading on the relevant common factor:
%
\begin{equation}
\log \sigma_{i,t}^2 =
\begin{cases}
\eta_{i,t}^{m} + h_{m,t} & \text{if variable } i \text{ is macroeconomic,} \quad i = 1, \ldots, n_m, \\[4pt]
\eta_{i,t}^{f} + h_{f,t} & \text{if variable } i \text{ is financial,} \quad i = 1, \ldots, n_f, \\[4pt]
\eta_{i,t}^{u} + h_{s_{i,t},\,t} \quad & \text{if variable } i \text{ is unclassified,} \quad i = 1, \ldots, n_u.
\end{cases}
\label{eq:variance}
\end{equation}
%
The idiosyncratic terms $\eta_{i,t}^{k}$ ($k \in \{m, f, u\}$) capture variable-specific volatility movements that are orthogonal to the common factors, ensuring that the log-volatilities of variables within the same block are not forced into perfect collinearity. Each $\eta_{i,t}^{k}$ follows a stationary AR(1) process, so that idiosyncratic volatility shocks are mean-reverting while the common factors $h_{m,t}$ and $h_{f,t}$ can exhibit the persistent swings characteristic of aggregate uncertainty episodes. Variables pre-classified as macroeconomic load exclusively on $h_{m,t}$, so that their common volatility component rises and falls only with aggregate real-economy uncertainty. Variables pre-classified as financial load exclusively on $h_{f,t}$. For unclassified variables, however, the common factor is selected by a discrete latent state indicator $s_{i,t} \in \{m, f\}$: when $s_{i,t} = m$, the variable's common volatility component is $h_{m,t}$; when $s_{i,t} = f$, it is $h_{f,t}$. This discrete switching mechanism---rather than a continuous blend of both factors---means that at each point in time the model assigns each unclassified variable entirely to one block, and the \textit{posterior probability} of that assignment becomes the object of inferential interest. As we explain in subsection~(1.2), this is the mechanism through which we identify whether a given external uncertainty source transmits to Taiwan through the macroeconomic channel or the financial channel.

The two common log-volatility factors evolve according to independent driftless random walks:
%
\begin{equation}
h_{k,t} = h_{k,t-1} + \zeta_{k,t}, \qquad \zeta_{k,t} \sim \mathcal{N}(0,\, \sigma_{\zeta,k}^2), \qquad k \in \{m, f\},
\label{eq:logvol}
\end{equation}
%
where the innovation variance $\sigma_{\zeta,k}^2$ controls the degree of time variation in factor $k$. The random-walk specification permits persistent, unbounded movements in the log-volatility factors, consistent with the empirical evidence that uncertainty exhibits prolonged episodes of elevation---such as the 2008--2009 global financial crisis or the 2018--2019 U.S.--China trade war---followed by gradual mean reversion \citep{bloom2014fluctuations}. Moreover, the random walk nests the constant-volatility case as a limiting special case when $\sigma_{\zeta,k}^2 \to 0$, so that the degree of time variation is estimated from the data rather than imposed by assumption. Because $h_{m,t}$ and $h_{f,t}$ are defined in logs, the implied level of uncertainty $\exp(\tfrac{1}{2} h_{k,t})$ is guaranteed to be positive at all times.

The final component governs the time-varying classification of unclassified variables, which is the mechanism that transforms the statistical model into an economic identification device. For each unclassified variable $i$, the latent state indicator $s_{i,t} \in \{\text{macro},\, \text{financial}\}$ in Equation~\eqref{eq:variance} follows a first-order two-state \textbf{Markov chain} with transition probabilities $q_{mm}^{(i)} = P(s_{i,t} = m \mid s_{i,t-1} = m)$ and $q_{ff}^{(i)} = P(s_{i,t} = f \mid s_{i,t-1} = f)$. The posterior classification probability
%
\begin{equation}
\pi_{i,t} = P(s_{i,t} = \text{macro} \mid \mathcal{F}_t)
\label{eq:pi}
\end{equation}
%
is the probability---conditional on the full information set $\mathcal{F}_t$---that unclassified variable $i$ belongs to the macroeconomic block at time $t$, updated recursively using the \citet{hamilton1989new} filter. This specification implies that a variable may be classified as macroeconomic during one episode and as financial during another, with the timing and persistence of regime switches determined entirely by the data. The transition probabilities $q_{mm}^{(i)}$ and $q_{ff}^{(i)}$ are estimated jointly with all other model parameters within the Bayesian MCMC framework, so that the persistence of each variable's classification is itself a model output rather than a researcher-imposed restriction. In our application, the time path of $\pi_{i,t}$ for each external variable constitutes the primary object of interest: it reveals, at each point in the sample, whether the uncertainty associated with a given external driver transmits to Taiwan predominantly through the macroeconomic channel ($\pi_{i,t} \approx 1$) or through the financial channel ($\pi_{i,t} \approx 0$).

The full model---comprising Equations~\eqref{eq:var}--\eqref{eq:pi}---is estimated jointly in a single-step Bayesian Markov chain Monte Carlo (MCMC) procedure developed by \citet{davidson2025investigating}. The algorithm builds on the computationally efficient precision-based sampler of \citet{cross2023large}, which exploits band and sparse matrix structures to draw the common log-volatilities in a single block, making estimation of very large SVMVARs feasible without resorting to dimensionality reduction techniques that would obscure the specific structural shocks we aim to identify. \citet{davidson2025investigating} extend this approach with a novel parameter transformation for sampling the unrestricted $\mathbf{B}_0$ and with the Markov-switching classification mechanism described above. We do not derive the prior distributions or the posterior sampling steps here; the complete Bayesian implementation is detailed in the online appendix of \citet{davidson2025investigating}. All subsequent analysis in this project---time-varying classification probabilities, forecast error variance decompositions, and impulse response functions---is conducted on draws from the joint posterior distribution, ensuring that parameter uncertainty and classification uncertainty are fully propagated through to the final inferential objects.

% Placeholder for remaining subsections
% (1.2) Variable Classification and the Key Innovation
% (1.3) Order-Invariance and Why It Matters for Taiwan
%
% (2) Anticipated problems and means of resolution
%   (2.1) Data Assembly and Preprocessing
%   (2.2) Works planned for the first year
