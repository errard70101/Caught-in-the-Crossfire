\documentclass[12pt]{article}
\usepackage{natbib}

\begin{document}

\section{Introduction}

Taiwan is deeply integrated with both the U.S. and Chinese economies, sitting directly in the path of major policy and geopolitical developments between the two countries. When the U.S. Federal Reserve tightens monetary policy, when trade disputes between the U.S. and China escalate, or when cross-strait political tensions rise, Taiwan often faces immediate consequences for its trade, investment, and financial flows.

However, major policy shifts and geopolitical realignments rarely move from initial proposal to a well-defined policy trajectory overnight. Central banks move in sequences of decisions, trade conflicts unfold through rounds of negotiation and retaliation, and cross-strait relations evolve over extended periods rather than at a single point in time. As a result, there is often a long interval between the announcement or anticipation of a policy change and the point at which a stable policy regime actually emerges. Along the way, planned policy changes may be revised, delayed, or even cancelled altogether. For example, rounds of U.S.--China tariff threats have repeatedly ended with smaller or postponed tariff increases than initially announced, and cross-strait initiatives have at times been shelved after domestic political opposition. During this interval, firms, households, and investors face heightened uncertainty about the future path of policy, demand, and financial conditions. This heightened uncertainty, in turn, affects their investment, consumption, and portfolio decisions \citep{bloom2009impact,bloom2014fluctuations}. One prominent example is trade policy uncertainty, which strongly influences firms' trade and investment decisions \citep{handley2022tpu}.

Empirically, however, identifying external uncertainty shocks and quantifying their effects on Taiwan presents several methodological challenges. First, external and domestic sources of uncertainty are tightly intertwined: developments in U.S.--China relations can simultaneously alter global risk sentiment, Taiwan's domestic political outlook, and expectations about cross-strait policy. This interconnection means that small-scale econometric models may suffer from severe omitted variable bias, potentially misattributing domestic uncertainty dynamics to external sources or vice versa \citep{carriero2018measuring}. Second, many existing approaches adopt a two-step procedure---first estimating an uncertainty measure, then assessing its macroeconomic effects in a separate model. This approach treats estimated uncertainty as observable data, introducing potential measurement error bias and model inconsistency \citep{carriero2018measuring}. Third, existing uncertainty measures, such as global financial volatility indices or country-specific policy uncertainty indices, do not distinguish between the \textit{channels} through which uncertainty transmits to the economy. For a small open economy like Taiwan, understanding whether external shocks propagate primarily through macroeconomic channels (trade flows, production linkages) or financial channels (capital flows, asset prices) carries distinct policy implications.

Against this background, this project addresses the following questions. First, do external uncertainty shocks from the United States and China transmit to Taiwan's economy primarily through \textit{macroeconomic channels}---affecting real activity, trade flows, and production---or through \textit{financial channels}---impacting asset prices, credit conditions, and capital flows? Second, which external sources---U.S. monetary policy, U.S.--China trade policy uncertainty, or broader geopolitical risks---contribute most to Taiwan's domestic economic uncertainty? Third, has the transmission mechanism changed over time, particularly during episodes of heightened U.S.--China tensions such as the 2018--2019 trade war or the post-2020 technology decoupling?

To address these questions, this project employs the order-invariant stochastic volatility in mean vector autoregression (OI-SVMVAR) framework developed by \citet{davidson2025investigating}. This framework offers three key advantages for our research objectives. First, it accommodates large-scale models (40+ variables), mitigating the omitted variable bias that plagues smaller models \citep{carriero2018measuring}. Second, its order-invariant specification ensures that results do not depend on arbitrary variable ordering---a critical feature for credible inference in large VAR systems. Third, its time-varying classification mechanism allows us to identify \textit{when} and \textit{how} external shocks shift between macroeconomic and financial transmission channels.

\bibliographystyle{aea}
\bibliography{research_proposal.bib}

\end{document}
