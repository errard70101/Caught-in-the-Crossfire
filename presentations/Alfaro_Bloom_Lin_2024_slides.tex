% LaTeX Beamer Presentation: The Finance Uncertainty Multiplier
% Alfaro, Bloom, and Lin (2024)
% Journal of Political Economy

\documentclass[aspectratio=169,11pt]{beamer}

% Theme and color settings
\usetheme{Madrid}
\usecolortheme{default}

% Packages
\usepackage{amsmath}
\usepackage{amssymb}
\usepackage{graphicx}
\usepackage{booktabs}
\usepackage{tikz}
\usepackage{multicol}

% Title information
\title[Finance Uncertainty Multiplier]{The Finance Uncertainty Multiplier}
\author[Alfaro, Bloom, Lin]{%
  Iv\'an Alfaro\inst{1} \and
  Nicholas Bloom\inst{2} \and
  Xiaoji Lin\inst{3}
}
\institute[]{
  \inst{1}BI Norwegian Business School \and
  \inst{2}Stanford University \and
  \inst{3}University of Minnesota
}
\date{Journal of Political Economy, 2024}

\begin{document}

% Title slide
\begin{frame}
  \titlepage
\end{frame}

% Table of contents
\begin{frame}{Outline}
  \tableofcontents
\end{frame}

%=============================================================================
\section{Introduction and Motivation}
%=============================================================================

\begin{frame}{Research Question}
\begin{block}{Central Question}
How do real and financial frictions \textbf{amplify, prolong, and propagate} the negative impact of uncertainty shocks?
\end{block}

\vspace{0.5cm}

\begin{itemize}
  \item Uncertainty shocks play significant role in economic crises (2008 financial crisis, COVID-19)
  \item \textbf{Empirical challenge}: Identifying causal effects of second-moment shocks
  \item \textbf{Theoretical challenge}: Generating large, persistent responses matching slow recovery
\end{itemize}

\vspace{0.3cm}

\begin{alertblock}{Key Insight}
Financial frictions substantially amplify uncertainty's impact by preventing firms from costlessly buffering shocks via financial channels
\end{alertblock}
\end{frame}

\begin{frame}{Motivation: Why Study Uncertainty and Finance Together?}

\begin{columns}[T]
\column{0.5\textwidth}
\textbf{Empirical Evidence:}
\begin{itemize}
  \item Sharp output drops during recent crises
  \item Slow recovery after Great Recession
  \item Correlation between uncertainty (VIX) and real/financial outcomes
  \item Financial conditions deteriorate during recessions
\end{itemize}

\column{0.5\textwidth}
\textbf{Theoretical Gaps:}
\begin{itemize}
  \item Real-only models: rapid bounce-back, no persistence
  \item Financial-only models: miss real-option effects
  \item Need interaction between both frictions
  \item Propagation to financial variables unexplained
\end{itemize}
\end{columns}

\vspace{0.5cm}

\begin{block}{This Paper's Contribution}
Shows that uncertainty shocks and financial shocks should NOT be considered individually, but \textbf{jointly} --- they amplify each other
\end{block}

\end{frame}

%=============================================================================
\section{Empirical Strategy}
%=============================================================================

\begin{frame}{Identification Challenge}

\begin{block}{The Endogeneity Problem}
\begin{itemize}
  \item Uncertainty measures (stock volatility) are endogenous
  \item First-moment effects (downturns) correlated with second-moment effects (uncertainty)
  \item When commodity prices fall, uncertainty rises simultaneously
  \item Standard approach (lagged uncertainty) insufficient
\end{itemize}
\end{block}

\vspace{0.3cm}

\begin{exampleblock}{Novel Solution: Differential Exposure Instrumentation}
Exploit firms' \textbf{differential industry-level exposure} to:
\begin{enumerate}
  \item Energy price volatility (oil)
  \item Exchange rate volatility (7 major currencies)
  \item Policy uncertainty (EPU)
\end{enumerate}
\end{exampleblock}

\end{frame}

\begin{frame}{Instrumentation Strategy: Key Intuition}

\textbf{Example: Oil Price Uncertainty}

\vspace{0.3cm}

\begin{center}
\begin{tabular}{lcc}
\toprule
\textbf{Firm Type} & \textbf{Directional Exposure} & \textbf{Uncertainty Exposure} \\
 & (First Moment) & (Second Moment) \\
\midrule
Mining/Oil exploration & Positive (+) & Positive (+) \\
Airlines & Negative (--) & Positive (+) \\
Business services & Zero (0) & Zero (0) \\
\bottomrule
\end{tabular}
\end{center}

\vspace{0.5cm}

\begin{block}{Key Separation}
Using absolute value of exposures $|\beta_j|$ multiplied by volatility shocks $\Delta\sigma$ allows us to:
\begin{itemize}
  \item \textbf{Control for} oil price level exposure (first moment)
  \item \textbf{Identify from} oil price uncertainty exposure (second moment)
\end{itemize}
\end{block}

\end{frame}

\begin{frame}{Construction of Instruments}

\textbf{Step 1: Estimate Industry-Level Sensitivities}

$$r_{i,t}^{\text{risk-adj}} = \alpha_j + \sum_c \beta_j^c \times r_t^c + \epsilon_{i,t}$$

where $\beta_j^c$ captures industry $j$'s sensitivity to commodity/currency/policy $c$

\vspace{0.5cm}

\textbf{Step 2: Construct Nine Instruments}

$$z_{i,t-1}^c = |\beta_{j,t-3}^c| \times \Delta\sigma_{t-1}^c$$

\begin{itemize}
  \item Absolute value $|\beta|$: nondirectional exposure (second-moment)
  \item $\Delta\sigma^c$: change in volatility of oil, 7 currencies, policy
  \item Time lag (3 years): ensures predetermined exposure
\end{itemize}

\end{frame}

\begin{frame}{Data and Sample}

\begin{block}{Data Sources}
\begin{itemize}
  \item \textbf{Stock returns}: CRSP daily data (1965--2019)
  \item \textbf{Accounting data}: Compustat annual (1965--2019)
  \item \textbf{Firm volatility}: 12-month standard deviation of daily returns
  \item \textbf{Option-implied volatility}: OptionMetrics (1996--2019)
  \item \textbf{Main 2SLS sample}: 1993--2019 (56,172 firm-year observations)
\end{itemize}
\end{block}

\vspace{0.3cm}

\begin{block}{Dependent Variables (Real \& Financial)}
\begin{itemize}
  \item \textbf{Real}: Investment rate, intangible investment, employment, COGS, sales
  \item \textbf{Financial}: Cash holdings, debt, corporate payout (dividends + repurchases)
\end{itemize}
\end{block}

\end{frame}

%=============================================================================
\section{Main Empirical Results}
%=============================================================================

\begin{frame}{Investment Results: OLS vs. 2SLS}

\begin{table}
\centering
\small
\begin{tabular}{lccccc}
\toprule
& \multicolumn{2}{c}{OLS Realized} & \multicolumn{2}{c}{IV Realized} & IV Implied \\
\cmidrule(lr){2-3} \cmidrule(lr){4-5} \cmidrule(lr){6-6}
Investment rate$_{i,t}$ & (1) & (2) & (3) & (4) & (5) \\
\midrule
$\Delta$Volatility$_{i,t-1}$ & $-0.023^{***}$ & $-0.024^{***}$ & $-0.057^{***}$ & $-0.041^{***}$ & $-0.058^{**}$ \\
 & (0.002) & (0.002) & (0.014) & (0.014) & (0.022) \\
\midrule
Firm controls & No & No & No & Yes & Yes \\
IV first-moment controls & No & No & Yes & Yes & Yes \\
Sample period & 1965--2019 & 1993--2019 & 1993--2019 & 1993--2019 & 1998--2019 \\
Observations & 95,394 & 56,172 & 56,172 & 56,172 & 26,977 \\
First-stage F-test &  &  & 87.22 & 79.68 & 69.91 \\
\bottomrule
\end{tabular}
\caption{Source: Alfaro, Bloom, and Lin (2024), Table 3}
\end{table}

\vspace{0.3cm}

\textbf{Key Findings:}
\begin{itemize}
  \item 2SLS estimates 1.7--2.4$\times$ larger than OLS
  \item Strong first-stage (F $>$ 70)
  \item Results robust to realized vs. implied volatility
\end{itemize}

\end{frame}

\begin{frame}{Effects on Real and Financial Variables}

\begin{center}
\small
\begin{tabular}{lcccc}
\toprule
& Investment & Intangible & Employment & Sales \\
& Rate & Investment & Growth & Growth \\
\midrule
$\Delta$Volatility$_{i,t-1}$ & $-0.041^{***}$ & $-0.052^{***}$ & $-0.032^{*}$ & $-0.217^{**}$ \\
 & (0.014) & (0.016) & (0.016) & (0.082) \\
\midrule
& Payout & Debt & Cash & \\
& Growth & Growth & Growth & \\
\midrule
$\Delta$Volatility$_{i,t-1}$ & $-0.423^{***}$ & $-0.137^{**}$ & $0.167^{**}$ & \\
 & (0.085) & (0.053) & (0.067) & \\
\bottomrule
\end{tabular}
\end{center}

\vspace{0.5cm}

\begin{block}{Key Patterns}
\begin{itemize}
  \item \textbf{Real variables}: Investment, employment, sales all decline
  \item \textbf{Financial variables}: Firms increase cash, cut debt and payouts
  \item Consistent with \textbf{precautionary savings} channel
\end{itemize}
\end{block}

\end{frame}

\begin{frame}{Amplification by Financial Frictions}

\textbf{Does uncertainty impact vary with financial conditions?}

\vspace{0.3cm}

\begin{center}
\small
\begin{tabular}{lcccc}
\toprule
Investment rate$_{i,t}$ & Baseline & + Fin Index & + Firm & Triple \\
 & & Interaction & Constraint & Interaction \\
\midrule
$\Delta\sigma_{i,t-1}$ & $-0.041^{***}$ & $-0.022^{**}$ & $-0.034^{***}$ & $-0.023^{**}$ \\
 & (0.014) & (0.010) & (0.012) & (0.010) \\
$\Delta\sigma_{i,t-1} \times$ Fin\_index$_t$ &  & $-0.023^{**}$ &  & $-0.019^{**}$ \\
 &  & (0.009) &  & (0.008) \\
$\Delta\sigma_{i,t-1} \times D_{i,t-5}^{\text{constrained}}$ &  &  & $-0.020^{***}$ & $0.002$ \\
 &  &  & (0.006) & (0.009) \\
$\Delta\sigma_{i,t-1} \times D_{i,t-5}^{\text{constrained}} \times$ Fin\_index$_t$ &  &  &  & $-0.014^{***}$ \\
 &  &  &  & (0.005) \\
\bottomrule
\end{tabular}
\end{center}

\vspace{0.3cm}

\begin{alertblock}{Amplification Effect}
Impact of uncertainty on constrained firms during high financial friction periods:
$$-0.023 - 0.019 + 0.002 - 0.014 = -0.054$$
This is \textbf{2.35$\times$} the baseline impact!
\end{alertblock}

\end{frame}

\begin{frame}{Time-Varying Impact: 2008-09 Financial Crisis}

\begin{center}
\includegraphics[width=0.75\textwidth]{figure2_placeholder}
\end{center}

\textbf{Key Findings from Time-Series Analysis:}
\begin{itemize}
  \item Mean impact normally around $-1.5$ percentage points
  \item During 2008--09: impact \textbf{tripled} to $-5.0$ percentage points
  \item Driven by both worsening credit conditions (Aaa-Baa spread) and binding firm constraints
  \item Explains why uncertainty is so damaging during recessions
\end{itemize}

\end{frame}

%=============================================================================
\section{Theoretical Model}
%=============================================================================

\begin{frame}{Model Overview}

\begin{block}{Key Features}
\textbf{Dynamic Stochastic General Equilibrium (DSGE) Model with:}
\begin{enumerate}
  \item Heterogeneous firms
  \item \textbf{Real frictions}: Fixed capital adjustment costs (Ss bands)
  \item \textbf{Financial frictions}: Costly external financing, cash management
  \item \textbf{Uncertainty shocks}: Time-varying volatility (macro \& micro)
  \item \textbf{Financial shocks}: Time-varying financing costs
\end{enumerate}
\end{block}

\vspace{0.3cm}

\begin{itemize}
  \item Firms maximize equity value by choosing investment and cash holdings
  \item Production function: $y_{j,t} = X_t z_{j,t} k_{j,t}^{\alpha}$ (decreasing returns)
  \item Stochastic productivity (aggregate $X_t$ and firm-specific $z_{j,t}$)
  \item Two-state Markov processes for volatilities and financing costs
\end{itemize}

\end{frame}

\begin{frame}{Firm's Problem}

\textbf{Value function:}
$$v_{j,t} = \max_{i_{j,t}, n_{j,t+1}} \left\{ e_{j,t} - \psi_{j,t} + E_t M_{t,t+1} v_{j,t+1} \right\}$$

\vspace{0.3cm}

\textbf{Key Components:}

\begin{itemize}
  \item \textbf{Payout before financing cost}: $e_{j,t} = y_{j,t} - i_{j,t} - h_{j,t} - g_{j,t}$
  \item \textbf{Real adjustment cost}: $g_{j,t} = c_k y_{j,t} \mathbf{1}_{\{i_{j,t} \neq 0\}}$ (nonconvex)
  \item \textbf{Financial adjustment cost}: $\psi_{j,t} = \eta_t |e_{j,t}| \mathbf{1}_{\{e_{j,t} < 0\}}$ (proportional)
  \item \textbf{Capital accumulation}: $k_{j,t+1} = (1-\delta)k_{j,t} + i_{j,t}$
  \item \textbf{Cash accumulation}: $n_{j,t+1} = R_n n_{j,t} + h_{j,t}$ (with $R_n < R$)
\end{itemize}

\vspace{0.3cm}

\begin{block}{Uncertainty Processes}
Macro and micro volatilities: $\sigma^X \in \{\sigma_L^X, \sigma_H^X\}$, $\sigma_{j,t}^z \in \{\sigma_L^z, \sigma_H^z\}$
\end{block}

\end{frame}

\begin{frame}{Policy Functions: Ss Bands and Financial Frictions}

\begin{center}
\includegraphics[width=0.85\textwidth]{figure3_placeholder}
\end{center}

\textbf{Key Insights from Policy Functions:}
\begin{itemize}
  \item Classic Ss band behavior: invest/inaction/disinvest regions
  \item \textbf{Financial frictions expand Ss bands} beyond real-only model
  \item \textbf{Uncertainty expands bands further} (real-option + cash-option effects)
  \item Second flat region appears with financial constraints (binding $E=0$)
  \item Triple interaction (high $\sigma$, high $\eta$, high $k$) creates largest inaction region
\end{itemize}

\end{frame}

%=============================================================================
\section{Model Results}
%=============================================================================

\begin{frame}{Calibration Strategy}

\begin{columns}[T]
\column{0.5\textwidth}
\textbf{Standard Parameters:}
\begin{itemize}
  \item Discount factor: $\beta = 0.988$
  \item Capital share: $\alpha = 0.70$
  \item Depreciation: $\delta = 0.05$
  \item Return on cash: $R_n = 0.97 R$
\end{itemize}

\column{0.5\textwidth}
\textbf{Matched Moments:}
\begin{itemize}
  \item Investment slope in IV regression
  \item Cash/revenue ratio
  \item Real adj. cost: $c_k = 0.03$
  \item Fin. costs: $\eta_L = 0.03$, $\eta_H = 0.06$
\end{itemize}
\end{columns}

\vspace{0.5cm}

\textbf{Uncertainty Calibration (following Bloom et al. 2018):}
\begin{itemize}
  \item Baseline: $\sigma_L^X = 0.0067$, $\sigma_L^z = 0.051$
  \item High state: $\sigma_H^X = 1.6 \times \sigma_L^X$, $\sigma_H^z = 4.1 \times \sigma_L^z$
  \item Transition: $\Pr(L \to H) = 0.026$, $\Pr(H \to H) = 0.943$ (shocks every 9.6 years)
\end{itemize}

\end{frame}

\begin{frame}{Main Quantitative Results}

\begin{table}
\centering
\begin{tabular}{lc}
\toprule
\textbf{Model Specification} & \textbf{Peak Drop in Output (\%)} \\
\midrule
Real frictions only & $-1.8$ \\
Real + Financial frictions & $-3.9$ \\
\midrule
\textbf{Amplification factor} & \textbf{2.17$\times$} \\
\bottomrule
\end{tabular}
\caption{Impact of Uncertainty Shocks (Table 1 from paper)}
\end{table}

\vspace{0.5cm}

\begin{block}{Three Key Model Predictions}
\begin{enumerate}
  \item \textbf{Amplification}: Financial frictions \textbf{double} the impact on output
  \item \textbf{Persistence}: Duration of drops doubles (1 period $\to$ 2+ periods)
  \item \textbf{Propagation}: Effects spread to financial variables (cash $\uparrow$, payouts $\downarrow$)
\end{enumerate}
\end{block}

\end{frame}

\begin{frame}{Impulse Responses: Real vs. Real+Financial Frictions}

\begin{center}
\includegraphics[width=0.9\textwidth]{figure4_placeholder}
\end{center}

\textbf{Comparing Models:}
\begin{itemize}
  \item \textbf{Real-only (black X)}: Sharp drop, rapid bounce-back, overshooting
  \item \textbf{Real+Financial (red triangle)}: Larger drop, persistent decline, slow recovery
  \item Financial model matches slow recovery after Great Recession
  \item TFP drops due to increased misallocation under uncertainty
\end{itemize}

\end{frame}

\begin{frame}{Propagation to Financial Variables}

\textbf{Model with Real Frictions Only:}
\begin{itemize}
  \item Investment drops, output falls
  \item Dividends \textbf{increase} (pay out profits)
  \item No cash dynamics
\end{itemize}

\vspace{0.5cm}

\textbf{Model with Real + Financial Frictions:}
\begin{itemize}
  \item Investment drops, output falls
  \item \textbf{Cash holdings increase} (precautionary savings)
  \item \textbf{Dividends and equity payouts fall}
  \item \textbf{Debt issuance falls}
  \item Matches empirical evidence in Table 4
\end{itemize}

%\vspace{0.5cm}

\begin{alertblock}{Key Mechanism}
Firms build cautionary cash balances after uncertainty shock, limiting internal funds available for investment rebound $\Rightarrow$ \textbf{persistence}
\end{alertblock}

\end{frame}

\begin{frame}{Robustness Checks}

\begin{center}
\includegraphics[width=0.7\textwidth]{figure5_placeholder}
\end{center}

\textbf{Alternative Specifications Tested:}
\begin{itemize}
  \item Different transition probabilities for financial shocks
  \item Model without cash
  \item Constant vs. proportional financial costs
  \item Nonconvex financial adjustment costs
  \item $\pm 10\%$ variations in real and financial adjustment cost parameters
\end{itemize}

\vspace{0.3cm}

\textbf{Conclusion}: Qualitative results robust across specifications

\end{frame}

%=============================================================================
\section{Policy Implications}
%=============================================================================

\begin{frame}{Policy Implications}

\begin{block}{1. Uncertainty and Finance are Complements, Not Substitutes}
\begin{itemize}
  \item Should NOT view as "uncertainty shocks vs. financial shocks"
  \item Instead: uncertainty shocks \textbf{amplified by} financial conditions
  \item Policy interventions should target \textbf{both} simultaneously
\end{itemize}
\end{block}

\vspace{0.3cm}

\begin{block}{2. Financial Interventions Most Effective During High Uncertainty}
\begin{itemize}
  \item Reducing financial frictions (e.g., credit facilities) has largest impact when uncertainty is elevated
  \item Explains rationale for emergency lending facilities (2008, 2020)
  \item Even small reductions in financing costs can have large effects
\end{itemize}
\end{block}

\vspace{0.3cm}

\begin{block}{3. Timing Matters: Act Early}
\begin{itemize}
  \item Effects are persistent (2+ quarters)
  \item Early intervention can prevent slow recovery dynamics
  \item Cash hoarding by firms limits investment rebound
\end{itemize}
\end{block}

\end{frame}

\begin{frame}{Recent Applications}

\textbf{COVID-19 Crisis (2020):}
\begin{itemize}
  \item Massive uncertainty spike (VIX $>$ 80)
  \item Simultaneous tightening of financial conditions
  \item Model predicts tripling of impact $\Rightarrow$ justifies aggressive policy response
  \item PPP loans, Fed facilities targeted at maintaining firm liquidity
\end{itemize}

\vspace{0.5cm}

\textbf{Silicon Valley Bank Collapse (2023):}
\begin{itemize}
  \item Banking sector shock creates financial friction spike
  \item Heightened uncertainty about financial stability
  \item Paper's framework suggests amplified effects on investment
  \item Supports case for rapid intervention to prevent cascade
\end{itemize}

\end{frame}

%=============================================================================
\section{Conclusion}
%=============================================================================

\begin{frame}{Summary of Contributions}

\begin{enumerate}
  \item \textbf{Empirical}: Novel instrumentation strategy addressing endogeneity
  \begin{itemize}
    \item Differential exposure to energy, currency, policy uncertainty
    \item Causal identification of second-moment effects
    \item Shows OLS underestimates by factor of 1.7--2.4
  \end{itemize}

  \vspace{0.3cm}

  \item \textbf{Empirical}: Amplification by financial frictions
  \begin{itemize}
    \item Constrained firms respond 60\% more than unconstrained
    \item Impact triples during financial crises (2008--09)
    \item Effects extend to real \textbf{and} financial variables
  \end{itemize}

  \vspace{0.3cm}

  \item \textbf{Theoretical}: DSGE model with real and financial frictions
  \begin{itemize}
    \item Amplification: doubles impact on output
    \item Persistence: doubles duration of drops
    \item Propagation: spreads effects to cash, debt, payouts
  \end{itemize}
\end{enumerate}

\end{frame}

\begin{frame}{Key Takeaways}

\begin{alertblock}{Main Result}
Financial frictions are a \textbf{multiplier} for uncertainty shocks:
\begin{itemize}
  \item Prevent costless buffering via financial channels
  \item Create cash-option effects complementing real-option effects
  \item Generate persistent responses via precautionary savings
\end{itemize}
\end{alertblock}

\vspace{0.5cm}

\begin{block}{For Policymakers}
\begin{itemize}
  \item Target uncertainty \textbf{and} financial conditions jointly
  \item Financial interventions most effective during high uncertainty periods
  \item Small reductions in financing costs can have large macroeconomic effects
\end{itemize}
\end{block}

\vspace{0.3cm}

\begin{block}{For Researchers}
\begin{itemize}
  \item Instrumentation strategy applicable to other uncertainty studies
  \item Framework useful for analyzing other shock interactions
  \item Importance of modeling both real and financial margins
\end{itemize}
\end{block}

\end{frame}

\begin{frame}[standout]
  \begin{center}
    {\Huge Thank You!}

    \vspace{1cm}

    {\large Questions and Discussion}

    \vspace{1.5cm}

    \small
    \textbf{The Finance Uncertainty Multiplier}\\
    Alfaro, Bloom, and Lin (2024)\\
    \textit{Journal of Political Economy}, Vol. 132, No. 2

    \vspace{0.5cm}

    Replication code: \texttt{https://nbloom.people.stanford.edu/research}
  \end{center}
\end{frame}

%=============================================================================
% Appendix (optional additional slides)
%=============================================================================

\appendix

\begin{frame}{Appendix: First-Stage Results}

\textbf{Nine Instruments:}
\begin{itemize}
  \item Oil price volatility exposure
  \item 7 major currency volatility exposures (EUR, CAD, JPY, GBP, CHF, AUD, SEK)
  \item Economic Policy Uncertainty (EPU) exposure
\end{itemize}

\vspace{0.3cm}

\textbf{First-Stage Performance:}
\begin{itemize}
  \item F-statistics: 70--87 (well above threshold of 10)
  \item Sargan-Hansen J-test: Cannot reject instrument validity (p $>$ 0.4)
  \item Instruments strong across all specifications
\end{itemize}

\vspace{0.3cm}

\textbf{Controls:}
\begin{itemize}
  \item 9 aggregate first-moment controls (price levels, not volatility)
  \item 2 firm-level controls (Tobin's Q, stock returns)
  \item 4 standard finance controls (tangibility, leverage, ROA, size)
  \item 8 lagged dependent variables (autocorrelation)
\end{itemize}

\end{frame}

\begin{frame}{Appendix: Model Details}

\textbf{Aggregate Productivity Process:}
$$\log(X_{t+1}) = \log(\bar{X})(1-\rho^X) + \rho^X \log(X_t) + \sigma_t^X \varepsilon_{t+1}^X$$

\textbf{Firm Productivity Process:}
$$\log(z_{j,t+1}) = \rho^z \log(z_{j,t}) + \sigma_{j,t}^z \varepsilon_{j,t+1}^z$$

\vspace{0.3cm}

\textbf{Stochastic Volatility (Two-State Markov):}
\begin{align*}
\sigma^X &\in \{\sigma_L^X, \sigma_H^X\}, \quad \Pr(\sigma_{t+1}^X = \sigma_l^X | \sigma_t^X = \sigma_k^X) = \pi_{k,l}^{\sigma^X} \\
\sigma_{j}^z &\in \{\sigma_L^z, \sigma_H^z\}, \quad \Pr(\sigma_{j,t+1}^z = \sigma_l^z | \sigma_{j,t}^z = \sigma_k^z) = \pi_{k,l}^{\sigma^z}
\end{align*}

\textbf{Financing Cost Shocks:}
$$\eta_t \in \{\eta_L, \eta_H\}, \quad \Pr(\eta_{t+1} = \eta_l | \eta_t = \eta_k) = \pi_{k,l}^{\eta}$$

\end{frame}

\end{document}
