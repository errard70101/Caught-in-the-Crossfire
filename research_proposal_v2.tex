\documentclass[12pt]{article}
\usepackage{natbib}

\begin{document}

\section{Introduction}

Taiwan is deeply integrated with both the U.S. and Chinese economies, sitting directly in the path of major policy and geopolitical developments between the two countries. When the U.S. Federal Reserve tightens monetary policy, when trade disputes between the U.S. and China escalate, or when cross-strait political tensions rise, Taiwan often faces immediate consequences for its trade, investment, and financial flows.

However, major policy shifts and geopolitical realignments, rarely move from initial proposal to a well-defined policy trajectory overnight. Central banks move in sequences of decisions, trade conflicts unfold through rounds of negotiation and retaliation, and cross-strait relations evolve over extended periods rather than at a single point in time. As a result, there is often a long interval between the announcement or anticipation of a policy change and the point at which a stable policy regime actually emerges. Along the way, planned policy changes may be revised, delayed, or even cancelled altogether. For example, rounds of U.S.–China tariff threats have repeatedly ended with smaller or postponed tariff increases than initially announced, and cross-strait initiatives have at times been shelved after domestic political opposition.  During this interval, firms, households, and investors face heightened uncertainty about the future path of policy, demand, and financial conditions. This heightened uncertainty, in turn, affects their investment, consumption, and portfolio decisions \citep{bloom2009impact,bloom2014fluctuations}. Especially, trade policy uncertainty has become a key source of uncertainty shocks, with significant implications for firms’ trade and investment decisions \citep{handley2022tpu}.

\bibliographystyle{aea}
\bibliography{research_proposal.bib}

\end{document}
