\documentclass[12pt,a4paper]{article}
\usepackage[utf8]{inputenc}
\usepackage{xeCJK}
\usepackage[margin=2.5cm]{geometry}
\usepackage{amsmath}
\usepackage{amssymb}
\usepackage{amsthm}
\usepackage{hyperref}
\usepackage{setspace}
\usepackage{graphicx}
\usepackage{booktabs}
\usepackage{enumitem}
\usepackage{indentfirst}
\usepackage{titlesec}

\setCJKmainfont{PingFang TC}

\onehalfspacing
\setlength{\parindent}{2em}

% Section formatting
\titleformat{\section}{\normalfont\large\bfseries}{\thesection.}{1em}{}
\titleformat{\subsection}{\normalfont\normalsize\bfseries}{(\arabic{subsection})}{1em}{}

\title{\textbf{「夾縫中的經濟體」:美中不確定性對台灣的時變傳導機制研究} \\[0.5em]
\large Caught in the Crossfire: Time-Varying Transmission of U.S.-China Uncertainty to Taiwan}

\author{計畫主持人:林士揚 \\
執行機構:國立東華大學經濟學系}

\date{}

\begin{document}

\maketitle

%===============================================================================
\section{研究計畫之背景}
%===============================================================================

\subsection{研究動機與問題}

台灣作為高度開放的小型經濟體,同時深度整合於美國與中國的經濟體系中,面臨獨特的外部不確定性暴露。當外部衝擊(如美國聯準會升息、美中貿易摩擦、中國信用緊縮)影響台灣時,這些衝擊究竟是透過\textbf{總體經濟管道}(影響出口、工業生產、就業)還是\textbf{金融管道}(影響資本流動、信用利差、匯率)傳導?更關鍵的是,此傳導機制是否隨時間改變?

本研究計畫應用 Davidson, Hou, and Koop (2025, 以下簡稱 DHK) 發展的\textbf{順序不變隨機波動均值向量自迴歸模型} (Order-Invariant Stochastic Volatility in Mean VAR, OI-SVMVAR),對上述問題提供嚴謹的實證分析。DHK (2025) 證明三個關鍵方法論問題困擾既有不確定性研究:(1) 模型規模重要——約30變數的小型模型產生偏誤估計;(2) 大型VAR中的變數排序造成順序依賴問題;(3) 研究者強制將變數分類為「總體」或「金融」可能不恰當。

\textbf{核心研究問題}:外部不確定性衝擊從美國與中國傳導至台灣時,主要透過總體經濟管道還是金融管道?此傳導機制如何隨時間變化?

\subsection{文獻回顧與研究缺口}

現有台灣不確定性研究存在重要方法論限制。Sin (2015) 運用六變數SVAR(四個台灣變數加兩個中國變數)研究中國經濟政策不確定性對台灣的影響,發現中國EPU顯著影響台灣產出與匯率,衝擊解釋約15\%的台灣產出預測誤差變異。然而此小規模方法可能遭受DHK大型模型架構所欲克服的遺漏變數偏誤。Huang et al. (2019) 建構台灣EPU指數,World Uncertainty Index (Ahir et al., 2022) 提供1956年起的季度不確定性資料,但均未採用能區分總體與金融傳導管道的結構識別方法。

國際文獻方面,Carriere-Swallow and Céspedes (2013) 研究不確定性衝擊對新興市場的影響,Brianti (2025) 證明總體不確定性衝擊引發通縮模式,允許央行同時穩定產出與通膨,而金融不確定性衝擊則需要政策取捨。對台灣央行而言,了解外部衝擊的傳導管道直接決定適當的政策工具選擇。

\textbf{本研究填補的文獻缺口}:
\begin{itemize}[noitemsep]
    \item 首次將「未分類變數」機制應用於識別外部衝擊傳導管道
    \item 首個針對台灣的大規模(43+變數)不確定性模型
    \item 首次量化美國、中國、美中關係三重暴露的相對重要性
    \item 提供時變傳導機制的實證證據
\end{itemize}

\begin{table}[htbp]
\centering
\caption{與既有台灣不確定性研究之比較}
\label{tab:comparison}
\small
\begin{tabular}{lcc}
\toprule
\textbf{面向} & \textbf{Sin (2015)} & \textbf{本研究} \\
\midrule
變數數目 & 6 & 43+ \\
台灣變數 & 4 & 28+ \\
中國變數 & 2 (EPU, IPI) & 3+ \\
美國變數 & 0 & 4+ \\
全球指標 & 0 & 4+ \\
\midrule
方法論 & SVAR & OI-SVMVAR \\
變數分類 & 固定 (研究者指定) & 時變 (資料驅動) \\
順序不變性 & 否 & 是 \\
隨機波動 & 否 & 是 \\
\midrule
研究問題 & 衝擊大小 & 傳導管道 \\
政策啟示 & 「中國重要」 & 「應對哪個管道」 \\
\bottomrule
\end{tabular}
\end{table}

\subsection{方法論創新:DHK (2025) 架構}

DHK (2025) 發展的OI-SVMVAR具有以下關鍵特徵:

\textbf{(a) 模型設定}

考慮 $n$ 維內生變數向量 $y_t$,基本模型為:
\begin{equation}
y_t = \sum_{i=1}^{p} B_i y_{t-i} + \sum_{j=0}^{q} A_j h_{t-j} + B_0^{-1} \epsilon_t^y
\label{eq:main}
\end{equation}
其中 $h_t = (h_{m,t}, h_{f,t})'$ 為二維潛在不確定性因子向量,分別代表總體經濟不確定性與金融不確定性。$B_0$ 為下三角結構矩陣,$\epsilon_t^y \sim N(0, \Omega_t)$,其中 $\Omega_t$ 為對角矩陣。

\textbf{(b) 變數分類與波動結構}

DHK的核心創新在於變數分類機制。將 $n$ 個變數分為三類:$n_m$ 個總體經濟變數、$n_f$ 個金融變數、$n_u$ 個未分類變數,其中 $n = n_m + n_f + n_u$。

對於第 $i$ 個變數,其對數波動 $\omega_{i,t}$ 的設定取決於其分類:

\textbf{總體經濟變數} ($i = 1, \ldots, n_m$):
\begin{equation}
\omega_{i,t}^m = \eta_{i,t}^m + h_{m,t}
\end{equation}

\textbf{金融變數} ($i = n_m+1, \ldots, n_m+n_f$):
\begin{equation}
\omega_{i,t}^f = \eta_{i,t}^f + h_{f,t}
\end{equation}

\textbf{未分類變數} ($i = n_m+n_f+1, \ldots, n$):
\begin{equation}
\omega_{i,t}^u = \eta_{i,t}^u + h_{s_{i,t},t}, \quad s_{i,t} \in \{m, f\}
\label{eq:unclassified}
\end{equation}
其中 $s_{i,t}$ 為離散潛在狀態變數,決定第 $i$ 個未分類變數在時間 $t$ 的分類歸屬,$\eta_{i,t}$ 為各變數的個別隨機波動成分。

\textbf{(c) 時變分類機制}

未分類變數的分類概率 $\pi_i = P(s_{i,t} = m)$ 由模型內生決定。DHK採用 Beta 先驗:
\begin{equation}
\pi_i \sim \text{Beta}(\underline{a}_\pi, \underline{b}_\pi)
\end{equation}
基準設定 $\underline{a}_\pi = \underline{b}_\pi = 1$ 對應均勻分佈,讓資料決定分類。

\textbf{(d) 共同對數波動的動態過程}

共同不確定性因子 $h_t$ 遵循 VAR(1) 過程:
\begin{equation}
h_t = \mu_h + \Phi_h (h_{t-1} - \mu_h) + \epsilon_t^h, \quad \epsilon_t^h \sim N(0, \Sigma_h)
\end{equation}
此設定允許總體與金融不確定性之間存在動態互動。

\textbf{(e) 順序不變性}

DHK的關鍵方法論貢獻是發展順序不變的MCMC演算法。傳統大型VAR採用下三角識別,結果依賴變數排序。DHK透過以下創新實現順序不變性:

\begin{enumerate}[noitemsep]
    \item 對 $B_0$ 採用對稱先驗結構
    \item 聯合抽樣所有波動狀態
    \item 對分類內變數進行對稱處理
\end{enumerate}

\textbf{(f) 本研究的創新應用}

本研究將DHK的「未分類變數」機制用於識別外部衝擊傳導管道,這是原始論文未探討的應用:

\begin{itemize}[noitemsep]
    \item \textbf{DHK的應用}:解決國內變數的分類模糊性(如S\&P 500對美國是總體還是金融指標?)
    \item \textbf{本研究的應用}:將\textit{所有外部衝擊來源}置於未分類類別,識別這些衝擊傳導至台灣的\textbf{管道}
\end{itemize}

當外部變數(如美國FFR)被模型歸類為「總體」($\pi_i \to 1$),表示該衝擊主要透過總體經濟管道傳導至台灣;若被歸類為「金融」($\pi_i \to 0$),則表示主要透過金融管道傳導。

%===============================================================================
\section{第一年:模型建構與資料準備}
%===============================================================================

\subsection{研究方法、進行步驟及執行進度}

\textbf{第一年工作重點}:建構台灣大規模資料集、實作DHK (2025) MCMC演算法、進行初步估計與驗證。

\subsubsection*{Step 1:資料收集與處理(第1-4個月)}

建構43+變數的月頻資料集,涵蓋1995年1月至2024年12月(360個觀測值):

\textbf{(A) 台灣總體經濟變數}(19個):
\begin{itemize}[noitemsep]
    \item 產出與經濟活動:工業生產指數、製造業生產指數、外銷訂單指數、實質零售銷售、實質出口、實質進口、製造業PMI、非製造業NMI、景氣對策信號分數
    \item 物價:CPI年增率、核心CPI年增率、躉售物價指數年增率、進口物價指數年增率、出口物價指數年增率
    \item 勞動市場:失業率、製造業就業人數、服務業就業人數、實質製造業薪資年增率
\end{itemize}

\textbf{(B) 台灣金融變數}(9個):
\begin{itemize}[noitemsep]
    \item 利率與利差:隔夜拆款利率、十年期公債殖利率、期限利差、信用利差
    \item 股票市場:加權指數月報酬率、日均成交量、月波動率、外資淨買超、融資餘額年增率
\end{itemize}

\textbf{(C) 未分類變數——外部衝擊來源}(11個):
\begin{itemize}[noitemsep]
    \item 美國變數:聯邦基金利率、工業生產指數年增率、BAA-AAA信用利差、經濟政策不確定性指數
    \item 中國變數:工業生產年增率、生產者物價指數年增率、社會融資規模年增率
    \item 全球指標:VIX、地緣政治風險指數(GPR)、全球經濟政策不確定性指數、美中貿易政策不確定性指數
\end{itemize}

\textbf{(D) 未分類變數——台灣國內模糊變數}(6個):
\begin{itemize}[noitemsep]
    \item 政策與貨幣:央行重貼現率、M1b年增率、M2年增率
    \item 資產價格:新台幣/美元匯率、加權指數水準、房價指數年增率
\end{itemize}

資料來源:主計總處、中央銀行、證交所、台灣經濟新報(TEJ)、FRED、CEIC。

\subsubsection*{Step 2:MCMC演算法實作(第3-6個月)}

實作DHK (2025) 的順序不變MCMC演算法,主要步驟包括:

\textbf{(a) 先驗設定}

VAR係數採用Minnesota型先驗:
\begin{equation}
\text{vec}(B) \sim N(\underline{b}, \underline{V}_B)
\end{equation}
其中 $\underline{V}_B$ 的設定遵循 Banbura et al. (2010) 的shrinkage原則。

波動過程參數:
\begin{align}
\mu_h &\sim N(\underline{\mu}_h, \underline{V}_{\mu_h}) \\
\text{vec}(\Phi_h) &\sim N(\underline{\phi}_h, \underline{V}_{\Phi_h}) \\
\Sigma_h &\sim IW(\underline{\nu}_h, \underline{S}_h)
\end{align}

分類概率:$\pi_i \sim \text{Beta}(1, 1)$(均勻先驗)。

\textbf{(b) MCMC抽樣步驟}

DHK演算法的核心抽樣步驟:

\begin{enumerate}
    \item \textbf{抽樣 $h_t$ 序列}:

    給定其他參數,$h_t$ 的條件後驗為非標準分佈。DHK採用 precision sampler:
    \begin{equation}
    p(h | y, \theta) \propto \exp\left( -\frac{1}{2} h' K_h h + k_h' h \right)
    \end{equation}
    其中 $K_h$ 為精確矩陣(precision matrix),$k_h$ 為相應向量,均由模型參數與資料決定。

    \item \textbf{抽樣分類狀態 $s_{i,t}$}:

    對每個未分類變數 $i$ 與時間 $t$:
    \begin{equation}
    P(s_{i,t} = m | \cdot) = \frac{\pi_i \cdot p(\omega_{i,t} | h_{m,t}, \eta_{i,t}^m)}{\pi_i \cdot p(\omega_{i,t} | h_{m,t}, \eta_{i,t}^m) + (1-\pi_i) \cdot p(\omega_{i,t} | h_{f,t}, \eta_{i,t}^f)}
    \end{equation}

    \item \textbf{抽樣分類概率 $\pi_i$}:

    給定分類狀態 $\{s_{i,t}\}_{t=1}^T$:
    \begin{equation}
    \pi_i | \{s_{i,t}\} \sim \text{Beta}\left(1 + \sum_{t=1}^T \mathbf{1}(s_{i,t}=m), 1 + \sum_{t=1}^T \mathbf{1}(s_{i,t}=f)\right)
    \end{equation}

    \item \textbf{抽樣 VAR 參數 $(B_i, A_j)$}:

    採用標準貝氏VAR方法,給定波動狀態進行條件抽樣。

    \item \textbf{抽樣波動過程參數 $(\mu_h, \Phi_h, \Sigma_h)$}:

    標準貝氏VAR估計,以 $h_t$ 序列為因變數。

    \item \textbf{抽樣個別波動 $\eta_{i,t}$}:

    採用 Kim, Shephard, and Chib (1998) 的混合常態近似法。
\end{enumerate}

\textbf{(c) 順序不變性的實現}

關鍵創新在於對 $B_0^{-1}$ 的處理。DHK採用:
\begin{equation}
B_0^{-1} = L D^{1/2}
\end{equation}
其中 $L$ 為單位下三角矩陣,$D$ 為對角矩陣。透過對 $L$ 採用對稱先驗並在MCMC中進行隨機重排列(random permutation),確保結果不依賴變數排序。

\subsubsection*{Step 3:模型驗證(第5-6個月)}

\begin{enumerate}[noitemsep]
    \item 使用DHK原始美國資料複製其結果,驗證程式正確性
    \item 進行收斂診斷:trace plots、Geweke診斷、有效樣本數計算
    \item 以台灣資料子集進行初步估計測試
\end{enumerate}

\subsection{預期可能遭遇之困難及解決途徑}

\textbf{困難1:資料可得性}

部分台灣變數可能無法取得完整的1995年起序列。

\textbf{解決途徑}:
\begin{itemize}[noitemsep]
    \item 優先選擇最長序列的關鍵變數
    \item 對較短序列的變數,評估是否可用代理變數或排除
    \item 必要時將樣本起始點調整至2000年
\end{itemize}

\textbf{困難2:計算複雜度}

DHK報告43變數模型單次估計約需30小時。

\textbf{解決途徑}:
\begin{itemize}[noitemsep]
    \item 申請國網中心高速計算資源
    \item 最佳化程式碼,利用矩陣運算加速
    \item 優先完成基準模型,穩健性檢驗依序進行
\end{itemize}

\textbf{困難3:MCMC演算法實作}

DHK的順序不變演算法涉及複雜的聯合抽樣。

\textbf{解決途徑}:
\begin{itemize}[noitemsep]
    \item 詳細研讀DHK論文及其補充材料
    \item 先以美國資料複製原結果,確認程式正確性
    \item 必要時聯繫原作者取得程式碼或技術建議
\end{itemize}

\subsection{預期完成之工作項目及成果}

\textbf{第一年預期成果}:

\begin{enumerate}
    \item \textbf{完整資料集}:43+變數、1995-2024年月頻資料庫,含完整文件說明
    \item \textbf{程式實作}:完成OI-SVMVAR的MCMC估計程式(MATLAB/R)
    \item \textbf{驗證報告}:成功複製DHK (2025)美國結果的技術報告
    \item \textbf{初步結果}:台灣資料的初步估計結果與收斂診斷
    \item \textbf{會議論文}:投稿國內經濟學術研討會(如台灣經濟學會年會)
\end{enumerate}

%===============================================================================
\section{第二年:實證分析與政策意涵}
%===============================================================================

\subsection{研究方法、進行步驟及執行進度}

\textbf{第二年工作重點}:完成完整估計、進行三步驟分析、發展政策意涵、撰寫論文。

\subsubsection*{Step 1:完整模型估計(第7-9個月)}

\textbf{(a) 基準模型估計}

執行完整43+變數OI-SVMVAR估計:
\begin{itemize}[noitemsep]
    \item MCMC:50,000次迭代,burn-in 25,000次,每5次保留1次
    \item 預計計算時間:約30小時
    \item 收斂診斷:Geweke統計量、trace plots、有效樣本數
\end{itemize}

\textbf{(b) 穩健性檢驗}

\begin{table}[htbp]
\centering
\small
\begin{tabular}{lcc}
\toprule
\textbf{模型規格} & \textbf{估計次數} & \textbf{預計時數} \\
\midrule
基準大型模型 (43變數) & 1 & 30 \\
小型模型比較 (30變數) & 1 & 15 \\
COVID前樣本 (1995-2019) & 1 & 30 \\
替代變數分類 & 2 & 60 \\
COVID虛擬變數設定 & 2 & 60 \\
\midrule
\textbf{總計} & \textbf{7} & \textbf{$\sim$195} \\
\bottomrule
\end{tabular}
\end{table}

\subsubsection*{Step 2:三步驟分析架構(第9-12個月)}

\textbf{分析步驟一:識別傳導管道}

工具:時變分類概率 $\pi_{i,t}$

對每個外部變數 $i$(美國FFR、中國IPI、VIX等),繪製其分類概率的時間序列:
\begin{equation}
\pi_{i,t} = P(s_{i,t} = m | \text{data})
\end{equation}

當 $\pi_{i,t} \to 1$:第 $i$ 個外部變數在時間 $t$ 主要透過\textbf{總體經濟管道}傳導至台灣。

當 $\pi_{i,t} \to 0$:主要透過\textbf{金融管道}傳導。

預期發現:
\begin{itemize}[noitemsep]
    \item 美國貨幣政策(FFR)在正常時期透過金融管道傳導,在全球衰退期轉為總體管道
    \item 中國經濟衝擊(IPI)主要透過總體管道傳導(貿易與供應鏈效應)
    \item VIX與GPR主要透過金融管道傳導(資本流動與風險溢價)
    \item 2018年美中貿易戰後,美中關係指標的傳導管道可能發生結構轉變
\end{itemize}

\textbf{分析步驟二:量化衝擊來源}

工具:預測誤差變異分解 (FEVD)

計算台灣國內不確定性被各外部來源解釋的比例:
\begin{equation}
\text{FEVD}_{h_{m,t}}^{(x_j)} = \frac{\text{Var}(h_{m,t+k} | x_j)}{\text{Var}(h_{m,t+k})}
\end{equation}

分析面向:
\begin{itemize}[noitemsep]
    \item 美國變數對台灣 $h_{m,t}$ 與 $h_{f,t}$ 的解釋比例
    \item 中國變數對台灣不確定性的貢獻
    \item 美中關係指標的相對重要性
    \item 全球風險指標(VIX, GPR)的影響
\end{itemize}

\textbf{分析步驟三:分析經濟後果}

工具:衝擊反應函數 (IRF)

計算台灣經濟變數對不確定性衝擊的動態反應:
\begin{align}
\frac{\partial y_{i,t+k}}{\partial h_{m,t}} &: \text{對總體不確定性衝擊的反應} \\
\frac{\partial y_{i,t+k}}{\partial h_{f,t}} &: \text{對金融不確定性衝擊的反應}
\end{align}

重點分析:
\begin{itemize}[noitemsep]
    \item 台灣工業生產對 $h_{m,t}$ vs. $h_{f,t}$ 衝擊的反應差異
    \item 股市與匯率對不同類型衝擊的反應模式
    \item 各歷史事件期間的反應差異:2008金融海嘯、2018貿易戰、2020疫情
\end{itemize}

\subsubsection*{Step 3:政策意涵發展(第12-14個月)}

基於實證結果,發展央行政策建議:

\textbf{情境1}:若外部衝擊主要透過金融管道傳導
\begin{itemize}[noitemsep]
    \item 優先採用匯率管理與資本流動監控
    \item 部署總體審慎工具(貸款成數限制、資本緩衝)
    \item 與金融監理機關協調
\end{itemize}

\textbf{情境2}:若外部衝擊主要透過總體管道傳導
\begin{itemize}[noitemsep]
    \item 聚焦傳統利率政策
    \item 與財政政策協調
    \item 支持產業結構調整政策
\end{itemize}

\textbf{情境3}:時變傳導機制
\begin{itemize}[noitemsep]
    \item 發展即時監控指標
    \item 設計狀態依存政策規則
    \item 建立政策工具快速切換的制度能力
\end{itemize}

\subsubsection*{Step 4:論文撰寫與發表(第14-18個月)}

\begin{itemize}[noitemsep]
    \item 完成完整研究論文(英文)
    \item 撰寫央行政策簡報(中文)
    \item 投稿國際學術期刊(目標:Journal of International Economics, Journal of Monetary Economics, 或 Journal of Applied Econometrics)
    \item 國際研討會發表
\end{itemize}

\subsection{預期可能遭遇之困難及解決途徑}

\textbf{困難1:COVID-19期間的極端值處理}

2020-2021年疫情期間產生極端異常值,可能主導估計結果。

\textbf{解決途徑}:
\begin{itemize}[noitemsep]
    \item 虛擬變數法:加入COVID啞變數(2020M2-2020M6急性期、2020M2-2021M12延長期)
    \item 穩健估計:對COVID期間觀測值進行1\%與99\%分位數縮尾處理
    \item 分樣本分析:分別估計COVID前(1995-2019)與全樣本(1995-2024)
    \item 詮釋架構:將COVID視為「前所未有的衝擊」,政策結論聚焦非COVID期間
\end{itemize}

\textbf{困難2:模型規模與計算時間權衡}

多規格估計總計約需195小時計算時間。

\textbf{解決途徑}:
\begin{itemize}[noitemsep]
    \item 優先完成基準模型與最關鍵的穩健性檢驗
    \item 利用高速計算資源平行處理獨立的估計任務
    \item 若時間不足,部分次要穩健性檢驗列為未來研究方向
\end{itemize}

\textbf{困難3:確保政策相關性}

須將技術性結果轉化為可操作的政策建議。

\textbf{解決途徑}:
\begin{itemize}[noitemsep]
    \item 研究過程中與央行研究人員保持溝通
    \item 以具體政策情境框架呈現結果
    \item 同時準備學術論文與政策簡報兩種產出
\end{itemize}

\subsection{預期完成之工作項目及成果}

\textbf{第二年預期成果}:

\begin{enumerate}
    \item \textbf{學術貢獻}:
    \begin{itemize}[noitemsep]
        \item 完成OI-SVMVAR台灣完整估計結果
        \item 首次識別外部衝擊傳導管道的時變特性
        \item 量化美國、中國、美中關係的相對重要性
        \item 發現模型規模對台灣不確定性研究的影響
    \end{itemize}

    \item \textbf{政策貢獻}:
    \begin{itemize}[noitemsep]
        \item 提供央行傳導管道特定的政策指引
        \item 建立外部衝擊監控優先順序
        \item 為匯率制度設計提供實證基礎
    \end{itemize}

    \item \textbf{具體產出}:
    \begin{itemize}[noitemsep]
        \item 國際期刊投稿論文1篇
        \item 央行政策簡報1份
        \item 國際研討會發表2-3場
        \item 可複製的程式碼與資料集(供後續研究使用)
    \end{itemize}
\end{enumerate}

%===============================================================================
\section{整合型研究計畫說明}
%===============================================================================

本計畫為個別型研究計畫,無整合型計畫架構。

未來可能的延伸方向包括:
\begin{itemize}[noitemsep]
    \item 將架構擴展至其他面臨雙重外部暴露的小型開放經濟體(韓國、新加坡、東協國家)
    \item 發展即時傳導管道監控系統
    \item 結合公司層級資料分析產業異質性
\end{itemize}

%===============================================================================
\section{參考文獻}
%===============================================================================

\begin{thebibliography}{99}

\bibitem{ahir2022}
Ahir, H., Bloom, N., \& Furceri, D. (2022). The World Uncertainty Index. \textit{NBER Working Paper No. 29763}.

\bibitem{baker2016}
Baker, S. R., Bloom, N., \& Davis, S. J. (2016). Measuring economic policy uncertainty. \textit{The Quarterly Journal of Economics}, 131(4), 1593--1636.

\bibitem{banbura2010}
Banbura, M., Giannone, D., \& Reichlin, L. (2010). Large Bayesian vector autoregressions. \textit{Journal of Applied Econometrics}, 25(1), 71--92.

\bibitem{brianti2025}
Brianti, M. (2025). Financial Shocks, Uncertainty Shocks, and Corporate Liquidity. \textit{Journal of Applied Econometrics} (forthcoming).

\bibitem{carriere2013}
Carriere-Swallow, Y., \& Céspedes, L. F. (2013). The impact of uncertainty shocks in emerging economies. \textit{Journal of International Economics}, 90(2), 316--325.

\bibitem{carriero2019}
Carriero, A., Clark, T. E., \& Marcellino, M. (2019). Large Bayesian vector autoregressions with stochastic volatility and non-conjugate priors. \textit{Journal of Econometrics}, 212(1), 137--154.

\bibitem{chan2020}
Chan, J. C. (2020). Large Bayesian VARs: A flexible Kronecker error covariance structure. \textit{Journal of Business \& Economic Statistics}, 38(1), 68--79.

\bibitem{davidson2025}
Davidson, J., Hou, C., \& Koop, G. (2025). Investigating economic uncertainty using stochastic volatility in mean VARs: The importance of model size, order-invariance and classification. \textit{Journal of Business \& Economic Statistics} (forthcoming).

\bibitem{huang2019}
Huang, Y.-F., Shih, P.-T., \& Wang, C.-W. (2019). Measuring economic policy uncertainty in Taiwan. \textit{Taiwan Economic Review}, 47(3), 361--401.

\bibitem{jurado2015}
Jurado, K., Ludvigson, S. C., \& Ng, S. (2015). Measuring uncertainty. \textit{American Economic Review}, 105(3), 1177--1216.

\bibitem{kim1998}
Kim, S., Shephard, N., \& Chib, S. (1998). Stochastic volatility: Likelihood inference and comparison with ARCH models. \textit{Review of Economic Studies}, 65(3), 361--393.

\bibitem{sin2015}
Sin, C.-Y. (2015). The economic effects of China's economic policy uncertainty on Taiwan. \textit{Taiwan Economic Forecast and Policy}, 46(1), 55--93.

\end{thebibliography}

\end{document}
