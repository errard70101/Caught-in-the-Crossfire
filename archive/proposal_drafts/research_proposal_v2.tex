\documentclass[12pt]{article}
\usepackage{natbib}

\begin{document}

\section{Introduction}

Taiwan is deeply integrated with both the U.S. and Chinese economies, sitting directly in the path of major policy and geopolitical developments between the two countries. When the U.S. Federal Reserve tightens monetary policy, when trade disputes between the U.S. and China escalate, or when cross-strait political tensions rise, Taiwan often faces immediate consequences for its trade, investment, and financial flows.

However, major policy shifts and geopolitical realignments, rarely move from initial proposal to a well-defined policy trajectory overnight. Central banks move in sequences of decisions, trade conflicts unfold through rounds of negotiation and retaliation, and cross-strait relations evolve over extended periods rather than at a single point in time. As a result, there is often a long interval between the announcement or anticipation of a policy change and the point at which a stable policy regime actually emerges. Along the way, planned policy changes may be revised, delayed, or even cancelled altogether. For example, rounds of U.S.–China tariff threats have repeatedly ended with smaller or postponed tariff increases than initially announced, and cross-strait initiatives have at times been shelved after domestic political opposition. During this interval, firms, households, and investors face heightened uncertainty about the future path of policy, demand, and financial conditions. This heightened uncertainty, in turn, affects their investment, consumption, and portfolio decisions \citep{bloom2009impact,bloom2014fluctuations}. One prominent example is trade policy uncertainty, which strongly influences firms' trade and investment decisions \citep{handley2022tpu}.

=== to be modified ===

Empirically, however, identifying external uncertainty shocks and their effects on Taiwan is challenging. First, it is difficult to separate shocks to the expected path of policies from shocks to current policy levels, even though both often move together in response to the same news. Second, external and domestic sources of uncertainty are tightly intertwined: developments in U.S.–China relations can simultaneously alter global risk sentiment, Taiwan’s domestic political outlook, and expectations about cross-strait policy. Third, existing measures of uncertainty, such as global financial volatility or U.S.-based policy uncertainty indices, may not fully capture how external shocks are transmitted to a small open economy like Taiwan.

Against this background, this project addresses the following questions. First, how do external monetary policy and trade policy uncertainty shocks affect Taiwan’s real activity, trade flows, and financial markets? Second, to what extent do these effects differ across sectors with different exposures to the United States and China? Third, has the transmission of external uncertainty to Taiwan changed over time, particularly since the escalation of U.S.–China tensions in the late 2010s?

=== to be modified ===

\bibliographystyle{aea}
\bibliography{research_proposal.bib}

\end{document}
