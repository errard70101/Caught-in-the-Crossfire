\documentclass[12pt]{article}
\usepackage{natbib}
\usepackage{xcolor}

\begin{document}

\section{Introduction}

Taiwan is deeply integrated with both the U.S. and Chinese economies, sitting directly in the path of major policy and geopolitical developments between the two countries. When the U.S. Federal Reserve tightens monetary policy, when trade disputes between the U.S. and China escalate, or when cross-strait political tensions rise, Taiwan often faces immediate consequences for its trade, investment, and financial flows.

However, major policy shifts and geopolitical realignments rarely move from initial proposal to a well-defined policy trajectory overnight. Central banks move in sequences of decisions, trade conflicts unfold through rounds of negotiation and retaliation, and cross-strait relations evolve over extended periods rather than at a single point in time. As a result, there is often a long interval between the announcement or anticipation of a policy change and the point at which a stable policy regime actually emerges. Along the way, planned policy changes may be revised, delayed, or even cancelled altogether. For example, rounds of U.S.--China tariff threats have repeatedly ended with smaller or postponed tariff increases than initially announced, and cross-strait initiatives have at times been shelved after domestic political opposition. During this interval, firms, households, and investors face heightened uncertainty about the future path of policy, demand, and financial conditions. This heightened uncertainty, in turn, affects their investment, consumption, and portfolio decisions \citep{bloom2009impact,bloom2014fluctuations}. One prominent example is trade policy uncertainty, which strongly influences firms' trade and investment decisions \citep{handley2022tpu}. \textcolor{red}{To capture these rapidly evolving dynamics, this project utilizes high-frequency monthly data, as quarterly data risks averaging out the very intervals of negotiation, threat, and revision that drive economic decisions.}

Empirically, however, identifying external uncertainty shocks and quantifying their effects on Taiwan presents several methodological challenges. First, external and domestic sources of uncertainty are tightly intertwined, creating a severe challenge for identification. A single geopolitical event, such as a shift in U.S.--China relations, often simultaneously triggers global financial volatility and shifts in Taiwan's domestic political landscape. In a small-scale model that lacks sufficient domestic controls, the economic impact of these internal political shifts would be erroneously attributed solely to the external shock. This form of omitted variable bias can lead to a misdiagnosis of the true drivers of uncertainty, potentially overstating the direct influence of external factors while neglecting local transmission mechanisms \citep{carriero2018measuring}. Addressing this issue requires expanding the information set to include a rich array of both domestic and external variables, necessitating the use of a large-scale econometric model. 

Second, while a large-scale model is necessary to address this bias, it introduces specific identification trade-offs. In large systems, standard identification methods based on Cholesky decomposition become problematic because the results are highly sensitive to the arbitrary ordering of variables, a conceptual flaw known as order dependence. To overcome this, an order-invariant framework is required. However, achieving order invariance is mathematically incompatible with the strict block exogeneity restrictions that prevent domestic variables from affecting global ones---a standard assumption in small open economy models. Consequently, instead of imposing these rigid constraints, we employ the data-driven identification strategy proposed by \citet{davidson2025investigating}, treating external drivers as ``unclassified variables''. This avoids arbitrary restrictions on the contemporaneous relationships between variables, allowing for a data-determined structure that can accommodate potential feedback loops, rather than ruling them out by assumption. This approach suits Taiwan's pivotal role in global technology supply chains, where strict zero restrictions might assume away significant feedback effects originating from the supply side.

Third, many existing approaches face two related limitations: they rely on proxy measures of uncertainty and often use a two-step procedure that first estimates uncertainty and then evaluates its macroeconomic effects in a separate model. Methodologically, the two-step design treats the estimated uncertainty series as observed data, ignores estimation uncertainty, and can induce measurement-error bias and model inconsistency \citep{carriero2018measuring}. Substantively, widely used text-based indices built from international news may not reflect domestic conditions. For example, during Nancy Pelosi’s 2022 visit, international coverage emphasized imminent war risk, while sentiment in Taiwan remained relatively calm. Such reliance on external proxies can create a ``perception gap'' and distort estimates of uncertainty’s impact on the local economy. To address these issues, we employ the proposed stochastic volatility in mean vector autoregression (SVMVAR) framework to jointly estimate uncertainty and its economic effects in a single step. By modeling uncertainty as a latent factor driven by the data itself, we avoid reliance on potentially misaligned external proxies, ensuring that our measure reflects actual domestic economic and financial conditions.






Finally, existing uncertainty measures, such as global financial volatility indices or country-specific policy uncertainty indices, do not distinguish between the \textit{channels} through which uncertainty transmits to the economy. Widely adopted indices such as the VIX or the \citet{baker2016measuring} Economic Policy Uncertainty index compress the multidimensional nature of uncertainty into a single scalar, obscuring whether a given shock propagates through real activity or financial markets. For a small open economy like Taiwan, this distinction is not merely academic: it carries fundamentally different policy implications. When external shocks operate primarily through \textbf{macroeconomic channels}---affecting trade flows, export demand, and production linkages---the appropriate policy response involves instruments oriented toward the real economy, such as export facilitation and structural adjustment support. When shocks instead propagate through \textbf{financial channels}---disrupting capital flows, asset prices, and credit conditions---the Central Bank of China (Taiwan) faces a different set of imperatives, including foreign exchange intervention and liquidity management. Existing empirical frameworks, however, are ill-equipped to make this distinction in a time-varying setting, leaving policymakers without a systematic basis for identifying which channel is dominant at any given point in time. This constitutes the fourth and most substantive methodological gap that the present project is designed to fill.

Against this background, this project addresses the following questions. First, do external uncertainty shocks from the United States and China transmit to Taiwan's economy primarily through \textit{macroeconomic channels}---affecting real activity, trade flows, and production---or through \textit{financial channels}---impacting asset prices, credit conditions, and capital flows? Second, which external sources---U.S. monetary policy, U.S.--China trade policy uncertainty, or broader geopolitical risks---contribute most to Taiwan's domestic economic uncertainty? Third, has the transmission mechanism changed over time, particularly during episodes of heightened U.S.--China tensions such as the 2018--2019 trade war or the post-2020 technology decoupling?

To address these questions, this project employs the order-invariant stochastic volatility in mean vector autoregression (OI-SVMVAR) framework developed by \citet{davidson2025investigating}. This framework offers three key advantages for our research objectives. First, it accommodates large-scale models (40+ variables), mitigating the omitted variable bias that plagues smaller models \citep{carriero2018measuring}. Second, its order-invariant specification ensures that results do not depend on arbitrary variable ordering---a critical feature for credible inference in large VAR systems. Third, its time-varying classification mechanism allows us to identify \textit{when} and \textit{how} external shocks shift between macroeconomic and financial transmission channels. By explicitly treating external drivers—such as U.S. monetary policy and cross-strait tension indices—as ``unclassified variables,'' we allow the model to endogenously determine their transmission path based on the data, transforming a statistical feature into a novel economic identification strategy.

This application constitutes a substantive methodological contribution that goes well beyond country replication. \citet{davidson2025investigating} deploy the unclassified variables device to resolve a question about the \textit{domestic} U.S. economy: which type of uncertainty---macroeconomic or financial---dominates over the business cycle? The present project repurposes the same device to answer an entirely different class of question, one that is uniquely relevant for small open economies: \textit{through which channel do external shocks enter the domestic economy?} This reorientation transforms a classification tool into a transmission-channel identifier. Because Taiwan's domestic uncertainty factors $h_{m,t}$ and $h_{f,t}$ remain anchored to clearly defined macroeconomic and financial variables, allowing external drivers to load endogenously onto these factors yields a direct, data-driven reading of the operative transmission mechanism at each point in time. The framework is, moreover, generalizable: any small open economy caught between competing large-economy influences---whether in Asia, Central Europe, or Latin America---can be analyzed within the same structure, suggesting a broad research agenda extending well beyond the Taiwan case.

The remainder of this proposal proceeds as follows. The first year of the project focuses on data assembly and empirical implementation: we construct a monthly dataset of more than 40 variables spanning Taiwan's domestic macroeconomic and financial conditions together with U.S., Chinese, and global indicators, adapt the \citet{davidson2025investigating} MCMC estimation algorithm to this dataset, and conduct the three-step analysis comprising time-varying classification probabilities, forecast error variance decomposition (FEVD), and impulse response functions (IRF). The second year extends the analysis by developing and estimating a small open economy dynamic stochastic general equilibrium (DSGE) model with financial frictions, estimated via Bayesian methods, whose structural impulse responses are matched against the data-driven IRFs obtained in year one, providing micro-founded validation of the transmission mechanisms identified empirically.

\bibliographystyle{aea}
\bibliography{research_proposal.bib}

\end{document}
